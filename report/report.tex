\documentclass{beamer}

\mode<presentation> {
    \usetheme{Madrid}
    \usecolortheme{rose}

    %\setbeamertemplate{footline}
    %若要删除所有幻灯片中的页脚,请取消注释此行

    %\setbeamertemplate{footline}[页码]
    %若要用简单的幻灯片计数替换所有幻灯片中的页脚,请取消注释此行

    %\setbeamertemplate{导航符号}{}
    %要删除所有幻灯片底部的导航符号,请取消注释此行
}

\usepackage{graphicx} % 允许包含图像
\usepackage{booktabs} % 允许在表中使用\toprule、\ midrule和\ bottomrule
\usepackage[UTF8,noindent]{ctexcap}  % 使用中文输入及显示
\usepackage[bookmarks=true]{hyperref}
\usefonttheme{serif}

%-----------------------------------
%	以下为正文
%-----------------------------------

\title[Scaling Laws]{
    Exploring Scaling Laws in LLM pretraining
}
% 简短标题显示在每张幻灯片的底部,完整标题仅在标题页上

\author{Zibo Ren,Runlin Chen} % Your name
\institute[PKU] % 您的机构将出现在每张幻灯片的底部,可能是节省空间的简写
{
    Peking University \\ % 你所在的机构
    \medskip
    \textit{2200010626@stu.pku.edu.cn,2200010848@stu.pku.edu.cn} % Your email address
}
\date{\today} % 日期,可以更改为自定义日期

\begin{document}

    \begin{frame}
        \titlepage % 将标题页打印为第一张幻灯片
    \end{frame}

    \begin{frame}
        \frametitle{Overview} % 目录幻灯片,注释此块以将其删除
        \tableofcontents % 在整个演示过程中,如果您选择使用\ section{}和\ submission{}命令,这些命令将自动打印在此幻灯片上,作为演示的概述
    \end{frame}

    %-----------------------------------
    %	开始创建PPT
    %-----------------------------------


    \section{Preliminary}\label{sec:preliminary}

    \begin{frame}

    \end{frame}

\end{document} 