\documentclass[aspectratio=169]{beamer}

\usepackage{graphicx}
\usepackage{booktabs}
\usepackage{subfigure}
\usepackage[bookmarks=true]{hyperref}
\usepackage{blkarray}
\usefonttheme{serif}

\mode<presentation> {
    \usetheme{Madrid}
    \usecolortheme{rose}

    \AtBeginSection[]{
        \begin{frame}{Overview}
            \tableofcontents[sectionstyle=show/shaded,subsectionstyle=show/shaded/hide]
        \end{frame}
    }

}

\title[Scaling Laws]{
    Exploring Scaling Laws in LLM Pretraining
}

\author{Zibo Ren, Runlin Chen}
\institute[PKU]
{
    Peking University \\
    \medskip
    \texttt{\{2200010626,2200010848\}@stu.pku.edu.cn}
}
\date{\today}

\begin{document}

    \begin{frame}
        \titlepage
    \end{frame}

    \begin{frame}
        \frametitle{Overview}
        \tableofcontents
    \end{frame}


    \section{Scaling Laws}\label{sec:scalinglaws}

    \subsection{Chinchilla Scaling Law}\label{subsec:chinchilla}
    \begin{frame}
        \frametitle{Chinchilla Scaling Law}
        \begin{block}{Chinchilla Scaling Law}
            \begin{equation}
                \label{eq:chinchilla}
                \begin{aligned}
                    L(D, N) &= L_0 + A\cdot D^{-\alpha} + B\cdot N^{-\beta} \\
                \end{aligned}
            \end{equation}
        \end{block}
        \begin{itemize}
            \item $L(D, N)$ is the final validation loss.
            \item $D$ is the amount of data.
            \item $N$ is the number of parameters.
            \item $L_0$, $A$, $B$, $\alpha$ and $\beta$ are undetermined
            positive constants.
        \end{itemize}

        It only illustrates the loss at the end of training, but not the
        loss during training.
    \end{frame}

    \subsection{Scaling Law with LR Annealing (LRA)}\label{subsec:LRA}

    \begin{frame}
        \frametitle{Scaling Law with LR Annealing}
        \begin{block}{Scaling Low Formula}
            \begin{equation}
                \label{eq:scaling_low}
                \begin{aligned}
                    L(s) &= L_0 + A\cdot S_1^{-\alpha} - C\cdot S_2 \\
                    S_1 &= \sum_{i=1}^{s} \eta_i \\
                    S_2 &= \sum_{i=1}^{s} \sum_{k=1}^{i} (\eta_{k-1} -
                    \eta_k)\cdot\lambda^{i-k}
                \end{aligned}
            \end{equation}
        \end{block}

        \begin{enumerate}
            \item $L(s)$: the loss at step $s$.
            \item $\eta_i$: the learning rate at step $i$.
            \item $\lambda$: a hyper-parameter to notate the decay factor
            in LR annealing momentum.
        \end{enumerate}
    \end{frame}

    \begin{frame}
        \frametitle{Scaling Law with LR Annealing}
        As an extension of Chinchilla scaling law, we can write the scaling law as
        \begin{equation}
            L(s) = L_0 + A\cdot S_1^{-\alpha} - f(\eta)
        \end{equation}
        The second term is an extension of the Chinchilla scaling law and
        the third term is a correction term for the learning rate decay.

        \textbf{Observation from Experiments:} there is a delay between
        learning rate
        changes and loss changes, so we write $f(\eta)$ as a form of momentum:
        \begin{equation}
            \begin{aligned}
                m_i &= \lambda m_{i-1} + (\eta_{i-1} - \eta_i) \\
                f(\eta) &= \sum_{i=1}^{s} m_i
            \end{aligned}
        \end{equation}
    \end{frame}

    \begin{frame}
        \frametitle{Defect of Scaling Law with LR Annealing}
        \begin{enumerate}
            \item If the we add some iterations with learning rate 0 to
            the end of the training, according to
            (\ref{eq:scaling_low}), the loss will still decrease dual
            to $S_1$ is not change and $S_2$ is increasing.
            \item Multiplying the entire schedule elementwise by a
            positive constant larger than 1 always decreases the predicted loss.
            If we set the learning rate to approach positive infinity
            during the first iteration and subsequently fix it at
            zero, this would theoretically yield a negative loss
            value, which clearly contradicts empirical observations.
            \item The form of $S_2$ is based on observation, but it is lack of
            theoretical support.
        \end{enumerate}
    \end{frame}

    \subsection{Multi-Power Law}\label{subsec:MPL}

    \begin{frame}
        \frametitle{Multi-Power Law}
        \begin{block}{Multi-Power Law}
            \begin{equation}
                \label{eq:multi_power_law}
                \begin{aligned}
                    &L(t) = L_0 + A\cdot (S_1(t) + S_W)^{-\alpha} - LD(t)\\
                    &\text{where} \\
                    & LD(t) = B\sum_{k=1}^{t}(\eta_{k-1}-\eta_k)\cdot
                    G(\eta_k^{-\gamma}S_k(t)), \\
                    &S_k(t) = \sum_{i=k}^{t} \eta_i, \\
                    &G(x) = 1-(Cx+1)^{-\beta}
                \end{aligned}
            \end{equation}
        \end{block}

        \begin{enumerate}
            \item $A\cdot (S_1(t) + S_W)^{-\alpha}$ is an extension of
            Chinchilla scaling law.
            \item $LD(t)$ is a correction term to account for the
            learning rate decay.
            \item Actually $G(x)$ can be any increasing function that
            maps $[0, \inf]$ to $[0, 1]$.
        \end{enumerate}
    \end{frame}

    \begin{frame}
        \frametitle{Multi-Power Law}
        \textbf{Observation:} If the sum of learning rates is identical, the
        loss at the end of training is similar, so we can write the loss
        as the following form:
        \begin{equation}
            L(t) = L_0 + A\cdot (S_1(t)-S_W)^{-\alpha} - LD(t)
        \end{equation}

        \textbf{Two-Stage Learning Rate Schedule:}
        \begin{enumerate}
            \item As $t$ increases, the difference between two learning
            rates schedulers is also increasing and eventually saturates.
            So we write $LD(T_A+x) = \tilde{B}(1-U(\eta_B x))$.
            \item According to experiments, $U(s) =
            (\tilde{C}s+1)^{-\beta}$ is a power-law form.
            \item $\tilde{B}$ is linear to the LR reduction, and $\tilde{C}$
            is a power-law of $\eta_B$.
        \end{enumerate}
        \begin{figure}
            \centering
            \includegraphics[width=0.4\textwidth]{fig/mpl.png}
        \end{figure}
    \end{frame}

    \begin{frame}
        \frametitle{Defect of Multi-Power Law}
        If the we add some iterations with learning rate 0
        to the training process, the loss will not change actually.
        For example, the learning rate list $[\eta_0,0,\eta_0,0]$ should
        get the same loss as $[\eta_0,\eta_0]$.
        But using MPL, we will get different loss values because
        the $G(x)$ term cannot cancel each other.
    \end{frame}


    \section{Experiments}\label{sec:experiments}

    \begin{frame}
        \frametitle{Experiments Setup}
        \begin{enumerate}
            \item We use the loss curves of a 100M GPT model trained on
            20B tokens of data.
            \item We use 3 types of learning rate schedules: "8-1-1",
            warmup-stable-decay(WSD) and cosine.
            \item The total train step is 33907, we use the first 10000
            of one learning rate schedule to fit the parameters of
            the model and test the model on the full loss curve of
            the three learning rate schedules.
            \item For both LRA and MPL, we use the SGD optimizer to fit
            the parameters.
            \item For MPL, we conducted a set of ablation experiments.The
            following methods are all
            modified from the original MPL: OPL means $LD(t)=0$, LLDL
            means $G(x)=1$ and MEL means $G(x) = 1-e^{-Cx},\gamma = 0$
        \end{enumerate}
    \end{frame}

    \begin{frame}
        \frametitle{Experiments Results of LRA}
        We use cosine learning rate schedule to fit the parameters
        \begin{figure}
            \centering
            \subfigure[cosine learning rate]{
                \includegraphics[width=0.3\textwidth]{fig/lra/cosine_fit.pdf}
            }
            \subfigure[8-1-1 learning rate]{
                \includegraphics[width=0.3\textwidth]{fig/lra/8-1-1_fit.pdf}
            }
            \subfigure[WSD learning rate]{
                \includegraphics[width=0.3\textwidth]{fig/lra/wsd_fit.pdf}
            }\label{fig:figure}
        \end{figure}
    \end{frame}

    \begin{frame}
        \frametitle{Experiments Results of MPL}
        We use cosine learning rate schedule to fit the parameters
        \begin{figure}
            \centering
            \subfigure[cosine learning rate]{
                \includegraphics[width=0.3\textwidth]{fig/mpl/cosine_fit.pdf}
            }
            \subfigure[8-1-1 learning rate]{
                \includegraphics[width=0.3\textwidth]{fig/mpl/8-1-1_fit.pdf}
            }
            \subfigure[WSD learning rate]{
                \includegraphics[width=0.3\textwidth]{fig/mpl/wsd_fit.pdf}
            }\label{fig:figure2}
        \end{figure}
    \end{frame}

    \begin{frame}
        \frametitle{Error of Fitting}

        The table below shows the fitting errors (MSE) of different scaling laws.
        \begin{table}[htbp]
            \centering
            \begin{tabular}{lcccc}
                \toprule
                Scaling Law & Schedule for Train & Cosine & 8-1-1
                & WSD \\
                \midrule
                LRA & Cosine & 4.4769e-02 &
                5.7181e-02 & 5.7885e-02 \\
                LRA & 8-1-1 & 4.5892e-02 &
                5.6336e-02 & 5.7780e-02 \\
                LRA & WSD & 4.4450e-02 &
                5.2244e-02 & 5.3246e-02 \\
                MPL & Cosine & 5.1094e-02 &
                6.1200e-02 & 5.9661e-02 \\
                MPL & 8-1-1 & 5.7103e-02 &
                5.5278e-02 & 5.3867e-02 \\
                MPL & WSD & 6.1385e-02 &
                5.2909e-02 & 5.1991e-02 \\
                OPL & Cosine & 5.1286e-02 &
                6.1284e-02 & 5.9737e-02 \\
                LLDL & Cosine & 5.1094e-02 &
                6.1200e-02 & 5.9661e-02 \\
                MEL & Cosine & 5.1115e-02 &
                6.1220e-02 & 5.9674e-02 \\
                \bottomrule
            \end{tabular}
            \caption{Fitting errors of different scaling laws}
            \label{tab:fitting_error}
        \end{table}

    \end{frame}

    \begin{frame}
        \frametitle{Phenomenon and Analysis}
        \begin{enumerate}
            \item[1] As shown in Table \ref{tab:fitting_error}, the
            fitting errors of all scaling laws under different
            training-test splits show minimal variation and remain
            below 0.1. This indicates that all scaling laws are
            effective in this set of experiments.
            \item[2] In the ablation experiments, we observe that
            modifications to the $LD(t)$ component result in
            negligible changes to fitting errors.
            This occurs because the learning rates in our
            experimental data vary smoothly, making this term several
            orders of magnitude smaller than the $S_1$ term, thereby
            exerting limited influence on the fitting outcomes.
        \end{enumerate}
    \end{frame}

    \begin{frame}
        \frametitle{Phenomenon and Analysis}
        \begin{enumerate}
            \item[3] Due to the minimal influence of the $LD(t)$ term,
            fitting curve of MPL closely approximates a simple power function.
            In contrast, fitting curve of LRA better captures the
            nuanced variations in loss (as illustrated in the figure
            below), thereby demonstrating superior fitting precision.
        \end{enumerate}
        \begin{figure}
            \centering

            \subfigure[LRA]{
                \includegraphics[width=0.4\textwidth]{fig/comparison/8-1-1_LRA}
            }
            \subfigure[MPL]{
                \includegraphics[width=0.4\textwidth]{fig/comparison/8-1-1_MPL}
            }
            \caption{Comparison of LRA and MPL fitting curves}\label{fig:figure3}
        \end{figure}
    \end{frame}

    \begin{frame}
        \frametitle{Optimal Learning Rate Schedule}
        Using MPL\eqref{eq:multi_power_law},we parameterize the predicted
        final loss as $\mathcal{L}_\Theta(E)$ with parameters $\Theta =
            {L_0,A,B,C,\alpha,\beta,\gamma}$ to get a optimization problem:
        $$
        \min_E \mathcal{L}_\Theta(E) \ \ s.t.
        \ \ 0\le\eta_t\le\eta_{t-1}, \ \forall t\in[1,T]
        $$
        Solving this problem,we can get the optimal learning rate
        schedule plotted in the figure below.
        \begin{figure}
            \centering
            \includegraphics[width=0.35\textwidth]{fig/opt/opt.pdf}
        \end{figure}

    \end{frame}


    \section{Simple Power Law}\label{sec:simplepowerlaw}

    \begin{frame}
        \frametitle{Discussion on Multi Power Law}
        Our group did not use the code provided by the original authors,
        but instead independently implemented the algorithm described in the paper.
        During our experiments, we observed that the parameters $B$, $C$,
        $\beta$, and $\gamma$ exhibited almost no variation throughout
        training. To investigate their actual impact, we performed a
        series of empirical evaluations.

        Our experiments revealed that manually setting a large value for
        $B$ allows the resulting model to fit the trend of the loss curve
        quite well. However, $C$ appeared to have virtually no effect
        during the experiments. The parameters $\beta$ and $\gamma$ only
        had an impact within certain ranges, and even then, their
        influence on the results was marginal.

        Moreover, although a larger $B$ helps the curve better track the
        loss trend visually, due to a shift in the mean, it does not
        necessarily lead to improved metrics such as MSE or $R^2$. This
        further confirms that the gradient with respect to $B$ is indeed
        very close to zero.

        In fact, the code provided by the paper’s authors is also unable
        to effectively optimize the parameter $B$. Their implementation
        selects $B$ from a set of hardcoded ``magic numbers'' rather than
        optimizing it in the conventional sense.
    \end{frame}

    \begin{frame}
        \frametitle{Discussion on Multi Power Law}
        There is a significant difference between the choice of $B$ in
        Multi Power Law
        \begin{figure}
            \centering
            \subfigure[MPL with $B=1$ fixed]{
                \includegraphics[width=0.4\textwidth]{fig/spl/mpl-1.pdf}
            }
            \subfigure[MPL with $B=80$ fixed]{
                \includegraphics[width=0.4\textwidth]{fig/spl/mpl-80.pdf}
            }
            \caption{Comparison of MPL with different $B$ values}
        \end{figure}
    \end{frame}

    \begin{frame}
        We fix $B$ = 80 in MPL and set $C$ to 1 or 100 sperately to observe the
        impact of $C$ on the fitting results. The results are shown in
        the following table.
        \begin{table}
            \centering
            \begin{tabular}{cccc}
                \toprule
                $C$ Value & MSE for cosine & MSE for WSD & MSE for 8-1-1 \\
                \midrule
                1         & 4.4730e-02     & 5.7266e-02  & 5.7939e-02    \\
                100       & 4.4732e-02     & 5.7267e-02  & 5.7939e-02    \\
                \bottomrule
            \end{tabular}\label{tab:table}
        \end{table}

        It can be seen that the value of $C$ has nearly no effect on the
        fitting results, which means the MPL model is over-parameterized
        and the parameter $C$ is not necessary.
    \end{frame}

    \begin{frame}
        \frametitle{Simple Power Law}
        In practice, simply setting $G(x) \equiv 1$ in MPL and assigning
        $B$ as a predefined parameter (just as the original paper) can
        already produce a good fit to the loss trend. We refer to this
        approach as the \textbf{Simple Power Law (SPL)}, which suggests
        that the trend of training loss in large models is, in a loose
        sense, proportional to the change in learning rate.

        \begin{figure}
            \centering
            \subfigure[MPL with $B = 80$]{
                \includegraphics[width=0.35\textwidth]{fig/spl/mpl-80.pdf}
            }
            \subfigure[SPL with $B=80$]{
                \includegraphics[width=0.35\textwidth]{fig/spl/spl.pdf}
            }
            \caption{Comparison of MPL and SPL}\label{fig:spl}
        \end{figure}
    \end{frame}

    \begin{frame}
        \frametitle{Simple Power Law}

        In our experiments, we found that the SPL can achieve nearly the
        same performance as the MPL while being significantly simpler.
        This suggests that sometimes a more complex scaling law with more
        parameters is not necessarily better, and simpler models can be
        more effective.
        \begin{table}
            \centering
            \begin{tabular}{cccc}
                \toprule
                Law & MSE($R^2$) for cosine & MSE($R^2$)  for WSD  & MSE($R^2$)  for 8-1-1 \\
                \midrule
                MPL & 4.4730e-02(0.9530)    & 5.7266e-02(0.9451) & 5.7939e-02(0.9446)  \\
                SPL & 4.4728e-02(0.9530)    & 5.7267e-02(0.9451) & 5.7939e-02(0.9446)  \\
                \bottomrule
            \end{tabular}\label{tab:table2}
        \end{table}
    \end{frame}

    \begin{frame}
        \begin{center}
            \Huge Thank You for Your Attention!
        \end{center}

        \begin{itemize}
            \item The division of labor: both group members participated in the code and slides, equal contribution.
            \item Our code and data are available at
            \url{https://github.com/0Ishtar0/explore-scaling-law.git}
        \end{itemize}
    \end{frame}

\end{document}